\documentclass[a4j,10pt]{jsarticle}
\usepackage{layout,url,resume}
\usepackage[dvipdfmx]{graphicx}
\pagestyle{empty}

\begin{document}
%\layout

\title{骨格推定を用いたボディビルのポージング練習ツールの提案}

% 和文著者名
\author{
    otot(田崎和輝)\thanks{NECO}
    \and
    親:ks91(斉藤賢爾) \thanks{NECO}
}

% 和文概要
\begin{abstract}
ここにアブストラクトを書く。
\end{abstract}

\maketitle
\thispagestyle{empty}

\section{はじめに}
近年、ボディビルを含むフィットネス業界は急速に発達しており、日本におけるフィットネスクラブの店舗数は2010年の約3600店舗から2018年では約5800店舗と1.6倍に増加し\cite{fitness}、
日本ボディビル・フィットネス連盟(JBBF)の登録選手数は2015年の2213人から2021年の5576人へと2倍位以上に増加している\cite{jbbf}。
しかしながら、ボディビル大会への出場は敷居が高く、トレーニング、減量だけでなくステージでの見栄えを良くするためにポージング練習も必須となる。
ポージング練習は初心者一人で行うのは難しく、トレーナーに指導を受けるという方法があるが高額である。

本研究では、骨格推定ライブラリであるOpenPoseを用いてカメラの入力から理想のポーズとの関節角度を比較し、視覚的にフィードバックを返すシステムを構築した。
\section{背景}
ボディビル競技とは日本ボディビル・フィットネス連盟によると

"
    競技としてのボディビルは、日頃のきびしいトレーニングで鍛え上げた全身の筋肉の発達度、そのダイナミックさ、美しさ、またバランスなどを競い合う個人スポーツです。\cite{bodybuilding}
"

審査は予選審査、決勝審査に分かれいる。それぞれの審査で規定の7ポーズをとる。決勝審査では規定のポーズに加え、音楽に合わせてポージングを行うフリーポーズ審査が行われる。審査基準は筋肉の大きさ(バルク)と形と明白さ(カット)、鮮明さ(デフィニション)、バランス(上下などの均斉)、ポーズの流れ、表現法などによる。

Openposeとはカーネギーメロン大学(CMU)の Zhe Caoら が「Realtime Multi-Person pose estimation」\cite{openpose}の論文で発表した、
深層学習を用いて人物のポーズを可視化してくれる手法であり、モーションキャプチャーなどの機器を使用することなく,
画像、動画データ、又はカメラからの入力を用いて人間のポーズを可視化することができる。



%---------------------------------------------

\section{問題と仮説}
\subsection*{問題}
既存の鏡の前で自分を見ながら行う練習方法では理想との差異が分かりにくい。全身を意識することが困難であるため、ある部分が修正されたときに他の部位が崩れてしまうということが起きてしまい、ポーズ習得に時間がかかってしまうと考える。
\subsection*{仮説}
理想のフォームとの差異をリアルタイムにフィードバックを行いながらポージングを練習することでポーズの習得時間を短縮できる。
\section{関連研究}
\subsection{hoge}
% 画像を図\ref{sample}に示す。

% % \begin{figure}[htbp]
% %     \begin{center}
% %         \includegraphics[width=6cm]{figure1.png}
% %         \caption{画像の例}
% %         \label{sample}
% %     \end{center}
% % \end{figure}
 
% \subsection{fuga}
fugafuga

\section{関連研究}

\section{提案手法}
\section{実験}
\section{評価}

\section{考察}

\bibliographystyle{junsrt}
\bibliography{resume}

\end{document}
% end of file
